\problemset{Формальные языки}

\graphicspath{{../1/}{../3/}}


\begin{problem}
Построить детерминированные \textbf{полные} конечные автоматы 
(если надо, добавить “дьявольскую” вершину) и 
представить описание в виде множества для 
следующих языков над алфавитом $\Sigma = \{a, b, c\}$:
\end{problem}

(a) каждое слово языка содержит подслово $bcc$ или $bba$;
$$
    L = \{\omega_1\alpha\omega_2 \mid \omega_1, \omega_2 \in \Sigma^*, \alpha=bcc \lor \alpha = bba \}
$$
\begin{figure}[h]
    \includegraphics[width=0.5\linewidth]{sm1.png}
    \caption{Конечный автомат для задачи 1 (a)}
\end{figure}

(b) слова заканчиваются всегда на другую букву, чем та, на которую они начинаются;
$$
    L = \{x \omega y \mid \omega \in \Sigma^*, x, y \in \Sigma, x \ne y \} \cup \{\epsilon\}
$$
\begin{figure}[h]
    \includegraphics[width=0.5\linewidth]{sm2.png}
    \caption{Конечный автомат для задачи 1 (b)}
\end{figure}

(c) буква $c$ не встречается “левее” буквы $b$ ни в одном слове
$$
    L = \{\omega_1\omega_2 \mid \omega_1 \in \{a, b\}^*,
    \omega_2 \in \{a, c\}^* \}
$$
\begin{figure}[h]
    \includegraphics[width=0.5\linewidth]{sm3.png}
    \caption{Конечный автомат для задачи 1 (c)}
\end{figure}

\begin{problem}
    Минимизировать каждый из автоматов в предыдущем задании. Если автомат уже
    минимален, доказать это.
\end{problem}

Каждый из автоматов минимален. 

\begin{proof}
    Детерминированный полный конечный автомат минимален $\Leftrightarrow$ 
    в нем все вершины достижимы и классы эквивалентности состояний 
    состоят из одного элемента, т.е. все состояния попарно неэквивалентны.

    Запись $q_1 \nsim_{\omega} q_2$ будет означать, 
    что строка $\omega$ различает состояния $q_1$ и $q_2$.  

    (a)
    \begin{align*}
        q0 &\nsim_{cc} \lq{b}\lq; q0 \nsim_{c} \lq bc \lq; q0 \nsim_{a} \lq bb \lq; q0 \nsim_{\epsilon} F \\
        \lq b \lq &\nsim_{c} \lq bc \lq; \lq b \lq \nsim_{a} \lq bb \lq; \lq b \lq \nsim_{\epsilon} F \\
        \lq bc \lq &\nsim_{c} \lq bb \lq; \lq bc \lq \nsim_{\epsilon} F \\
        \lq bb \lq &\nsim_{\epsilon} F
    \end{align*}

    (b)
    \begin{align*}
        q_0 &\nsim_{\epsilon} q_1, q_3, q_5; q_0 \nsim_{c} q_2, q_4; q_0 \nsim_{a} q_6   \\
        q_1 &\nsim_{\epsilon} q_2, q_4, q_6; q_1 \nsim_{a} q_3, q_5 \\
        q_2 &\nsim_{\epsilon} q_3, q_5; q_2 \nsim_{a} q_4, q_6 \\
        q_3 &\nsim_{\epsilon} q_4, q_6; q_3 \nsim_{b} q_5 \\
        q_4 &\nsim_{\epsilon} q_5; q_4 \nsim_{b} q_6 \\
        q_5 &\nsim_{\epsilon} q_6
    \end{align*}
    (c)
    \begin{align*}
        q_0 &\nsim_{b} q_1; q_0 \nsim_{\epsilon} q_2 \\
        q_1 &\nsim_{\epsilon} q_2
    \end{align*}
\end{proof}

\begin{problem}
    Реализовать алгоритм минимизации конечных автоматов.
\end{problem}

Программная реализация:

\url{https://github.com/bagar0x60/automata-hw2}
\break

Примеры работы скрипта:

\begin{figure}[h]
    \centering
    \begin{subfigure}[b]{0.4\linewidth}
      \includegraphics[width=\linewidth]{test2.png}
      \caption{Исходный}
    \end{subfigure}
    \begin{subfigure}[b]{0.4\linewidth}
      \includegraphics[width=\linewidth]{test2_minimized.png}
      \caption{Минимизированный}
    \end{subfigure}
    \caption{Пример 1 - удаление недостижимых состояний}
\end{figure}

\begin{figure}[h]
    \centering
    \begin{subfigure}[b]{0.6\linewidth}
      \includegraphics[width=\linewidth]{test1.png}
      \caption{Исходный}
    \end{subfigure}
    \begin{subfigure}[b]{0.3\linewidth}
      \includegraphics[width=\linewidth]{test1_minimized.png}
      \caption{Минимизированный}
    \end{subfigure}
    \caption{Пример 2 - Объеденение стоковых состояний}
\end{figure}

\begin{figure}[h]
    \centering
    \begin{subfigure}[b]{0.6\linewidth}
      \includegraphics[width=\linewidth]{test3.png}
      \caption{Исходный}
    \end{subfigure}
    \begin{subfigure}[b]{0.3\linewidth}
      \includegraphics[width=\linewidth]{test3_minimized.png}
      \caption{Минимизированный}
    \end{subfigure}
    \caption{Пример 3 - преобразование автомата без кончных состояний}
\end{figure}

\begin{figure}[h]
    \centering
    \begin{subfigure}[b]{0.8\linewidth}
      \includegraphics[width=\linewidth]{test5.png}
      \caption{Исходный}
    \end{subfigure}
    \begin{subfigure}[b]{0.6\linewidth}
      \includegraphics[width=\linewidth]{test5_minimized.png}
      \caption{Минимизированный}
    \end{subfigure}
    \caption{Пример 4}
\end{figure}