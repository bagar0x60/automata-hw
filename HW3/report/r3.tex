\problemset{Формальные языки}

\graphicspath{{../1/}{../2/}}
\newcommand\eqdef{\mathrel{\overset{\makebox[0pt]{\mbox{\normalfont\tiny\sffamily def}}}{=}}}

\begin{problem}
    Найти описание лексического синтаксиса вашего второго самого любимого языка программирования. В нем найти описание следующих языков:
\end{problem}

Для сокращение записи, определим следующие синтаксические конструкции
для регулярных выражений:

\begin{align*}
    A+ &\eqdef AA* \\
    A? &\eqdef A|\epsilon \\
    [a_1-a_n] &\eqdef a_1 | a_2 | ... | a_n
\end{align*}

Если мы захотим использовать один из специальных символов 
`+`, `*`, `?`, `|`, `(`, `)`, `[`, `]` в качестве одного из символов нашего языка, 
то мы будем экранировать этот символ обратным слешем: `$\backslash$+`, `$\backslash$*', ...  

$\bullet$ язык идентификаторов;

Описание: \url{https://doc.rust-lang.org/grammar.html#identifiers}

Регулярное выражение: $([a-z]|[A-Z]|\_)([0-9]|[a-z]|[A-Z]|\_)*$

\begin{figure}[h]
    \centering
    \begin{subfigure}[b]{0.4\linewidth}
      \includegraphics[width=\linewidth]{ident_undetermenized.png}
      \caption{Недетерминированный}
    \end{subfigure}
    \begin{subfigure}[b]{0.4\linewidth}
      \includegraphics[width=\linewidth]{ident.png}
      \caption{Детерминированный минимальный}
    \end{subfigure}
    \caption{Конечные автоматы для языка идентификаторов}
\end{figure}

\break

$\bullet$ язык каких-нибудь чисел (целых в десятичной счисления, с плавающей точкой,
любых других).

Описание: \url{https://doc.rust-lang.org/grammar.html#number-literals}

Опишем только десятичные числа с плавающей точкой 
(прим. $42, 12.34, 1.79E2, 100\_000\_000$ и т.д.) 

Регулярное выражение: $[0-9]([0-9]|\_)*(.([0-9]|\_)+)?((e|E)(-|\backslash+)?([0-9]|\_)+)?$

\begin{figure}[h]
    \centering
    \begin{subfigure}[b]{0.8\linewidth}
      \includegraphics[width=\linewidth]{numbers_undetermenized.png}
      \caption{Недетерминированный}
    \end{subfigure}
    \begin{subfigure}[b]{0.8\linewidth}
      \includegraphics[width=\linewidth]{numbers.png}
      \caption{Детерминированный минимальный}
    \end{subfigure}
    \caption{Конечные автоматы для языка идентификаторов}
\end{figure}

\break

$\bullet$ язык ключевых слов (можно какого-нибудь их конечного подмножества);

Описание: \url{https://doc.rust-lang.org/grammar.html#keywords}

Регулярное выражение: 
\begin{align*}
    &\_|abstract|alignof|as|become|box|\\
    &break|const|continue|crate|\\
    &do|else|enum|extern|false|\\
    &final|fn|for|if|impl|\\
    &in|let|loop|macro|match|\\
    &mod|move|mut|offsetof|override|\\
    &priv|proc|pub|pure|ref|\\
    &return|Self|self|sizeof|static|\\
    &struct|super|trait|true|type|\\
    &typeof|unsafe|unsized|use|virtual|\\
    &where|while|yield 
\end{align*}

\begin{figure}[h]
    \centering
    \begin{subfigure}[b]{0.4\linewidth}
      \includegraphics[width=\linewidth]{keywords_undetermenized.png}
      \caption{Недетерминированный}
    \end{subfigure}
    \begin{subfigure}[b]{0.4\linewidth}
      \includegraphics[width=\linewidth]{keywords.png}
      \caption{Детерминированный минимальный}
    \end{subfigure}
    \caption{Конечные автоматы для языка ключевых слов}
\end{figure}

\break

\begin{problem}
    Реализовать алгоритм Томпсона детерминизации конечных автоматов.
\end{problem}

Программная реализация:

\url{https://github.com/bagar0x60/automata-hw}
\break

Примеры работы скрипта:

\begin{figure}[h]
    \centering
    \begin{subfigure}[b]{0.6\linewidth}
      \includegraphics[width=\linewidth]{test0.png}
      \caption{Исходный}
    \end{subfigure}
    \begin{subfigure}[b]{0.6\linewidth}
      \includegraphics[width=\linewidth]{test0_determinized.png}
      \caption{Детерминированный}
    \end{subfigure}
    \caption{Пример детерминизации 1}
\end{figure}

\begin{figure}[h]
    \centering
    \begin{subfigure}[b]{0.4\linewidth}
      \includegraphics[width=\linewidth]{test1.png}
      \caption{Исходный}
    \end{subfigure}
    \begin{subfigure}[b]{0.4\linewidth}
      \includegraphics[width=\linewidth]{test1_determinized.png}
      \caption{Детерминированный}
    \end{subfigure}
    \caption{Пример детерминизации 2}
\end{figure}

\begin{figure}[h]
    \centering
    \begin{subfigure}[b]{0.3\linewidth}
      \includegraphics[width=\linewidth]{test2.png}
      \caption{Исходный}
    \end{subfigure}
    \begin{subfigure}[b]{0.6\linewidth}
      \includegraphics[width=\linewidth]{test2_determinized.png}
      \caption{Детерминированный}
    \end{subfigure}
    \caption{Пример детерминизации 2}
\end{figure}

\begin{figure}[h]
    \centering
    \begin{subfigure}[b]{0.4\linewidth}
      \includegraphics[width=\linewidth]{test3.png}
      \caption{Исходный}
    \end{subfigure}
    \begin{subfigure}[b]{0.4\linewidth}
        \includegraphics[width=\linewidth]{test3_no_eps.png}
        \caption{Без $\epsilon$-переходов}
    \end{subfigure}
    \begin{subfigure}[b]{0.6\linewidth}
      \includegraphics[width=\linewidth]{test3_determinized.png}
      \caption{Детерминированный}
    \end{subfigure}
    \caption{Пример детерминизации 3}
\end{figure}